#Presentazioni con LaTeX

\documentclass{beamer} 
% parte di latex che crea le presentazioni
\usepackage{listings}
%
\usepackage{color} 
%per definire tutti i colori

% Temi per presentazioni https://mpetroff.net/files/beamer-theme-matrix/
% Righe = strutture
% Colonne = colori

\usetheme{Berkeley} % Frankfurt, Madrid
%  Struttura tema
\usecolortheme{spruce} %dove, crane
%  Colore tema

\title{La mia prima presentazione}
\author{Pamela Sbardella}

\begin{document}
\maketitle

%crea una slide automatica (outline) prima di ogni sezione
\AtBeginSection[] % Do nothing for \section*
{	
\begin{frame}
\frametitle{Outline}
\tableofcontents[currentsection,currentsubsection,currentsubsubsection]
\end{frame}
}



\section{Introduzione}

\begin{frame}
\frametitle{My first slide!}
\begin{itemize}  % per creare un elenco
\item Mars' surface is covered with \textbf{craters} that formed when comets and/or asteroids impacted the red planet.  %\textbf{craters} per parole in grassetto
\item Mars' surface is covered with craters that formed when comets and/or asteroids impacted the red planet.
\item \pause These craters are especially numerous in the\textbf{southern highlands} of Mars   %\pause fa partire la scritta in ritardo (animazione)
\end{itemize} 
\end{frame}


\begin{frame}
\frametitle{Instruments used}
\centering
% \includegraphics[width=0.5\textwidth]{logo_unibo.png}
\includegraphics[width=0.7\textwidth]{cinghiale.jpg}  %devi prima caricare l'immagine su overleaf
\end{frame}



\section{Metodi}

\begin{frame}
\frametitle{Instruments used}
\centering
% \includegraphics[width=0.5\textwidth]{logo_unibo.png}
\includegraphics[width=0.7\textwidth]{dream.jpg}
\end{frame}



\begin{frame}
\frametitle{The used formula}
\centering
\includegraphics[width=0.3\textwidth]{dream.jpg}\\
\begin{equation}
e^{i \pi} + 1 = 0
\end{equation}
\end{frame}



\section{Risultati}

\begin{frame}
\frametitle{Instruments used}
\centering
% slide con 2 immagini vicine
\includegraphics[width=0.4\textwidth]{cinghiale.jpg} 
\includegraphics[width=0.4\textwidth]{dream.jpg}
\end{frame}


\begin{frame}
\frametitle{Instruments used}
\centering
% slide con 4 immagini vicine
\includegraphics[width=0.4\textwidth]{cinghiale.jpg} 
\includegraphics[width=0.4\textwidth]{dream.jpg} \\ %va a capo
\bigskip %(aumenta lo spazio a capo) oppure \smallskip
\includegraphics[width=0.4\textwidth]{cinghiale.jpg} 
\includegraphics[width=0.4\textwidth]{dream.jpg}
\end{frame}


\end{document}
