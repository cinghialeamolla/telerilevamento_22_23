% Da R a Latex

% R -> funzione(argom1, argom2, ...)
% #commento 

% Latex -> \funzione[struttura dell'argomento(non sempre presente)]{argomento}
% %commento





%%%  CODICE LATEX




\documenclass[a4paper,12pt]{article}
% è la classe utilizzata (come raster, dataframe, matrix, ecc), che indica il tipo di documento da fare
% [struttura del testo: foglio A4, dimensione carattere 12]
% {article = articolo}

\usepackage[utf8]{inputenc}
% è la funzione per caricare i pacchetti (come "library" in R)
% [definisce delle strutture - "utf8" sono tutti i tipi di carattere usabili in "inputenc"] 
% {definisce accenti e lettere particolari}

\usepackage{graphicx} % pacchetto per parti grafiche

\usepackage{color}
% può cambiare il colore del testo

\title{My first document in LaTeX}
% indica il titolo del documento

\author{Pamela Sbardella}
% indica l'autore del documento

\date{}
% si può scrivere manualmente [May 2022], lasciare le parentesi vuote e farla scomparire, o metterla come commento (% \date{}) per farla prendere in automatico dal pc dall'ultima ricompilazione

\begin{document}
% è la funzione che fa iniziare qualcosa (alla fine deve esserci sempre una funzione "end)
% qui sotto si definisce tutto quello che sarà all'interno del documento

\maketitle
% questa funzione pesca titolo e autore (definiti esternamente, sopra) e li carica

% abstract
% keywords
% double line

\section{Introduction}
% indicano le "sezioni"
%qui si introduce il "problema"; alla fine della sezione va esplicitato lo scopo del documento

% esempio di introduzione
%\textcolor{red}{aerial photography} -> cambia il colore della parte di testo tra parentesi
% \newcommand{\tr}{\textcolor{red}}  -> associa il nome "tr" alla funzione \textcolor{red}, che può essere poi usata nel testo; se commentato, automaticamente torna nero tutto il testo dopo "tr"

Since the 1920s, \textcolor{red}{aerial photography} has represented an \tr{important data} source for the detection of landscape patterns and their change over time.
%inserendo a fine riga due \\ e andando a capo si crea uno spazio tra le due frasi _ oppure utilizzare la funzione \smallskip (che lascia uno spazio più corto) o \bigskip (che lo lascia più ampio)

%andando a capo e lasciando uno spazio, LaTeX andrà a capo con un'indentazione; altrimenti usare la funzione \noindent davanti
Multitemporal analysis represents a powerful method for the study of all ecological and geological processes that change over time. The literature involves several fields of study: from soil loss to natural resources assessment to vegetation and ecological dynamics.

%subsection

\section{Study area}

The study area is the nature reserve of Poggio all’Olmo in Tuscany, Italy (11 28 26E, 42 51 45N, WGS84 Datum). It is located on the side of Mt. Amiata, comprising 440 ha, with elevations ranging from 650 to 1016 m above mean sea level (m.s.l.) and slopes from 0 to 55.

\section{Methods}
% si descrivono tutti i metodi utilizzati per realizzare il lavoro

\section{Results}
% si inseriscono tutti i risultati ottenuti in uscita (tabelle, figure, ecc)

\section{Discussion}
% si discutono tutti i risultati, alla luce dei risultati di altri scienziati a livello mondiale

\section{Conclusion}
% conclusione finale

\section{References}
% citazioni/bibliografia




\end{document}
% indica la fine di quello che era iniziato con la funzione "begin"
