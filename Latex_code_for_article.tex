% Da R a Latex

% R -> funzione(argom1, argom2, ...)
% #commento 

% Latex -> \funzione[struttura dell'argomento(non sempre presente)]{argomento}
% %commento
%   \%  %il \ davanti alla percentuale la "protegge" e la mantiene, anzichè inserire un commento
% \newpage funzione che porta a capo il testo



%%%  CODICE LATEX




\documenclass[a4paper,12pt]{article}
% è la classe utilizzata (come raster, dataframe, matrix, ecc), che indica il tipo di documento da fare
% [struttura del testo: foglio A4, dimensione carattere 12]
% {article = articolo}

\usepackage[utf8]{inputenc}
% è la funzione per caricare i pacchetti (come "library" in R)
% [definisce delle strutture - "utf8" sono tutti i tipi di carattere usabili in "inputenc"] 
% {definisce accenti e lettere particolari}

\usepackage{graphicx} % pacchetto per parti grafiche

\usepackage{color}
% può cambiare il colore del testo

\usepackage{hyperref}
% per fare i collegamenti link con le figure di riferimento (con la funzione ref); per figure e bibliografia

%\usepackage{lineno}
% inserisce i numeri di riga (per articoli scientifici)
%\linenumbers %funzione sul pacchetto lineno

\usepackage{listings}
% serve per inserire parti di codice

\usepackage{natbib}
%per inserire la bibliografia


\title{My first document in LaTeX}
% indica il titolo del documento

\author{Pamela Sbardella}
% indica l'autore del documento

\date{}
% si può scrivere manualmente [May 2022], lasciare le parentesi vuote e farla scomparire, o metterla come commento (% \date{}) per farla prendere in automatico dal pc dall'ultima ricompilazione

\begin{document}
% è la funzione che fa iniziare qualcosa (alla fine deve esserci sempre una funzione "end)
% qui sotto si definisce tutto quello che sarà all'interno del documento

\maketitle
% questa funzione pesca titolo e autore (definiti esternamente, sopra) e li carica

\tableofcontents  % inserisce un indice

% abstract
\begin{abstract}
Here is the abstract of my Master Thesis.

The Master Thesis is dealing with something.

Concluding, I made beautiful science.
\end{abstract}

% keywords
% double line

\section{Introduction}
% indicano le "sezioni"
%qui si introduce il "problema"; alla fine della sezione va esplicitato lo scopo del documento

% esempio di introduzione
%\textcolor{red}{aerial photography} -> cambia il colore della parte di testo tra parentesi
% \newcommand{\tr}{\textcolor{red}}  -> associa il nome "tr" alla funzione \textcolor{red}, che può essere poi usata nel testo; se commentato, automaticamente torna nero tutto il testo dopo "tr"

Since the 1920s, \textcolor{red}{aerial photography} has represented an \tr{important data} source for the detection of landscape patterns and their change over time.
%inserendo a fine riga due \\ e andando a capo si crea uno spazio tra le due frasi _ oppure utilizzare la funzione \smallskip (che lascia uno spazio più corto) o \bigskip (che lo lascia più ampio)

%andando a capo e lasciando uno spazio, LaTeX andrà a capo con un'indentazione; altrimenti usare la funzione \noindent davanti
Multitemporal analysis represents a powerful method for the study of all ecological and geological processes that change over time. The literature involves several fields of study: from soil loss to natural resources assessment to vegetation and ecological dynamics.

%subsection

\section{Study area}

The study area is the nature reserve of Poggio all’Olmo in Tuscany (Figure \ref{fig:cinghiale}), Italy (11 28 26E, 42 51 45N, WGS84 Datum). It is located on the side of Mt. Amiata, comprising 440 ha, with elevations ranging from 650 to 1016 m above mean sea level (m.s.l.) and slopes from 0 to 55.


%per caricare un'immagine bisogna salvarla sul pc, caricarla su overleaf e poi inserire il codice seguente:
\begin{figure}
\centering  % per centrare l'immagine
\includegraphics[width=0.4\textwidth]{cinghiale.jpg}  % funzione per inserire la figura; width indica la dimensione (0.4 volte la dimensione del testo)
\caption{This is my guide animal.}  % didascalia immagine
\label{fig:cinghiale}   % da un'etichetta alla figura, che la richiama con la funzione "ref" inserita sopra nel testo
\end{figure}


\section{Methods}
% si descrivono tutti i metodi utilizzati per realizzare il lavoro

In this section I am going to describe the methods of this manuscript. First of all I will strat with the formulas used and then I will pass to the code!
 
\subsection{Formulas}  % inserisce una sottosezione
Here is the formula used in this manuscript:


\begin{equation}   % per inserire una formula matematica  (https://en.m.wikibooks.org/wiki/LaTeX/Mathematics)
 F = G \times \frac{m_{1} \times m_{2}}{d^{2}}
\label{eq:newton}
\end{equation}

Let's make a more complex equation

\begin{equation}
 F = \sqrt[3]{G \times \frac{m_{1} \times m_{2}}{d^{2}}}
\label{eq:galileo2}
\end{equation}
 
\begin{equation}
F = \frac{\sqrt[3]{G \times \frac{m_{1} \times m_{2}}{d^{2}}}}{\sum{p(x) \times \log{p(x)}}}
\label{eq:galileo3}
\end{equation}

\begin{equation}
F = G \times \frac{m_{1} \times m_{2}}{d^{2 \times \mu}}  %per inserire una lettera greca
\label{eq:newton2}
\end{equation}

% alpha= biodiversità locale (in un punto preciso); beta= b tra un punto e un altro; gamma=b totale
% lambda= simbolo lunghezze d'onda (rosso, infrarosso, ecc)
 
\subsection{Code}
Here is the code used in this manuscript:
% per inserire del codice serve il pacchetto "listing" e bisogna caricare il codice su un file di testo, su overleaf, con la seguente formula [qui va il tipo di linguaggio usato]
\lstinputlisting[language=R]{R_code.r.txt}  % txt va bene (blocco note), rtf no.



\section{Results}
% si inseriscono tutti i risultati ottenuti in uscita (tabelle, figure, ecc)

As a result of this study I found that the Cadmium is present in the analyzed soil with a total amount of 15\%.
Let's put a formula directly in the main text. We can apply this: $F=G \times m_{1}$. These results were achieved according to Equation \ref{eq:newton}.



\section{Discussion}
% si discutono tutti i risultati, alla luce dei risultati di altri scienziati a livello mondiale

\section{Conclusion}
% conclusione finale

\section{References}
% citazioni/bibliografia




\end{document}
% indica la fine di quello che era iniziato con la funzione "begin"
